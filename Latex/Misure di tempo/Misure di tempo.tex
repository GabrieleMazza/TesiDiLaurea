\documentclass[a4paper,12pt]{report}							% COMANDI INIZIALI
\usepackage[italian]{babel}								% sillabazione italiana
\usepackage[utf8]{inputenc}								% Per le lettere accentate IN UNIX E IN WINDOWS
\usepackage{ragged2e}					 				% giustifica
\usepackage{amsmath}									% Per allineare le equazioni
\usepackage{amssymb}									% Per le lettere dell'indicatrice (mathbb)
\usepackage{bm}										% Per le lettere matematiche in grassetto (vettori)

\justifying 										% giustifica

\date{28 Luglio 2014}
\author{Gabriele Mazza}
\title{Dimostrazione $S_{time}$ e $S_{space}$}

\begin{document}

%Indice e numerazione
\pagenumbering{arabic}

\chapter{Misure di tempo}

Quando si crea un codice per una analisi complessa occorre tener conto delle scelte computazionali nell'implementazione, in quanto possono influenzare in modo molto rilevante l'esecuzione. Nel caso di analisi senza covariate l'operazione più importante e complessa è senza dubbio la stima della soluzione:
$$
\underline{\hat{c}}=(\Pi^T\Pi+S)^{-1}\Pi^T\underline{z}
$$
Occorre scegliere in modo accurato la struttura dati per le matrici e il metodo per la risoluzione del sistema, avendo come obiettivo la minimizzazione del tempo di esecuzione. Infatti, per stimare $\underline{\hat{c}}$ il sistema lineare da risolvere ha dimensione $NM$, valore elevato in alcuni dei casi analizzati.

Il codice è stato implementato su R, tutta




\end{document}