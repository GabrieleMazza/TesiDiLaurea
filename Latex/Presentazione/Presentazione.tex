\documentclass[landscape,9pt]{beamer}                           % COMANDI INIZIALI
\usepackage[italian]{babel}                             % sillabazione italiana
\usepackage[utf8]{inputenc}                             % Per le lettere accentate IN UNIX E IN WINDOWS
\usepackage{ragged2e}                                   % giustifica
\usepackage{amsmath}                                    % Per allineare le equazioni
\usepackage{amssymb}                                    % Per le lettere dell'indicatrice (mathbb)
\usepackage{graphicx}                                   % Per le figure
\usepackage{bm}
\usepackage{subfigure}
\usepackage{animate}
\usepackage[export]{adjustbox}
\usepackage{array,multirow,graphicx}
\usepackage[misc,geometry]{ifsym}
%\usepackage{cm-super}

\renewcommand{\fontsubfuzz}{1.1pt}                          % Elimina i warning inutili

\justifying                                         % giustifica

\usetheme{CambridgeUS}
\date{29 Aprile 2015}
\author{Gabriele Mazza}
\title{Regressione con regolarizzazioni differenziali per dati spazio-temporali, con applicazione all'analisi della produzione di rifiuti urbani nella provincia di Venezia}

\makeatletter
\setbeamertemplate{footline}
{
  \leavevmode%
  \hbox{%
  \begin{beamercolorbox}[wd=.5\paperwidth,ht=2.25ex,dp=1ex,center]{author in head/foot}%
    \usebeamerfont{author in head/foot}\insertshortauthor\expandafter\beamer@ifempty\expandafter{\beamer@shortinstitute}{}{~~(\insertshortinstitute)}
  \end{beamercolorbox}%
  \begin{beamercolorbox}[wd=.5\paperwidth,ht=2.25ex,dp=1ex,right]{date in head/foot}%
    \usebeamerfont{date in head/foot}\insertshortdate{}\hspace*{2em}
    \insertframenumber{} / \inserttotalframenumber\hspace*{2ex} 
  \end{beamercolorbox}}%
  \vskip0pt%
}
\makeatother
\setbeamercolor{date in head/foot}{use=frametitle, bg=frametitle.bg}
\setbeamercolor{subsection in head/foot}{use=framtitle, bg=frametitle.bg}

\begin{document}

\begin{frame}
\maketitle
\begin{center}
\includegraphics[width=0.25\textwidth,
	height=0.28\textheight]
	{Immagini/Logo.png}
\end{center}
\end{frame}

\section{Introduzione}
\begin{frame}
\frametitle{Introduzione}
In questo lavoro di tesi è costruito e analizzato il modello di \textit{Regressione Spazio-Temporale con Penalizzazioni Differenziali} (ST-PDE) per dati distribuiti in spazio e tempo:
$$
z=f(\bm{p},t)
$$
con $\bm{p} \in \Omega$, $t \in [T_1,T_2]$.
\ \ 
\newline
\newline
Grande attenzione sarà dedicata al dominio spaziale.
\end{frame}

\subsection{Applicazione alla produzione di rifiuti urbani nella provincia di Venezia}
\begin{frame}
L'applicazione scelta riguarda l'analisi della produzione di rifiuti urbani nella provincia di Venezia tra il 1997 e il 2011
\begin{figure}[h]
	\centering
	\subfigure
   {
	\includegraphics[width=0.43\textwidth]{Immagini/CTQDA.png}   
   }
	\subfigure
   {
	\includegraphics[width=0.49\textwidth]{Immagini/andamenti_temporali.png}
   }
\end{figure}
\end{frame}



\section{Presentazione modello ST-PDE}
\subsection{Caso senza covariate}
\begin{frame}
\frametitle{Presentazione modello ST-PDE}
Definisco:
\begin{itemize}
\item $\{\bm{p}_i = (x_i,y_i); i=1, \ldots , n\} \subset \Omega \subset \mathbb{R}^2$
\item $\{t_j ; j=1, \ldots , m\} \subset [T_1,T_2]\subset \mathbb{R}$
\item $z_{ij}$ osservazioni in $(\bm{p}_i,t_j)$
\end{itemize}
\ \ 
\newline
Modello:
$$
z_{ij}=f(\bm{p}_i,t_j)+\varepsilon_{ij}\ \ \ \ i = 1,\ldots,n\ \ j=1,\ldots,m \ \ 
$$
\ \ 
\newline
$\varepsilon_{ij}$ rumore iid di media nulla e varianza $\sigma^2$
\end{frame}

\begin{frame}
Funzioni di base in spazio e tempo:
\begin{eqnarray*}
\{ \varphi_k(t);k = 1, \ldots , M \} &\qquad & \mbox{ basi temporali definite in } [T_1,T_2] \\
\{ \psi_l(\bm{p});l = 1, \ldots , N \} &\qquad & \mbox{ basi spaziali definite in } \Omega
\end{eqnarray*}
\begin{figure}[t]
	\centering
	\subfigure
	{
	\includegraphics[width=0.46\textwidth]{Immagini/elementofinito.jpg}  
   }
	\subfigure
   {
	\includegraphics[width=0.46\textwidth]{Immagini/Bsplines.png}
   }
\end{figure}
\end{frame}

\begin{frame}
La funzione è espressa tramite le funzioni di base:
$$
f(\bm{p}, t) = \sum_{k=1}^M a_k(\bm{p})\varphi_k(t) = \sum_{l=1}^N b_l(t)\psi_l(\bm{p}) = \sum_{l=1}^N \sum_{k=1}^M c_{lk}\psi_l(\bm{p})\varphi_k(t)
$$
La soluzione si ricaverà minimizzando il funzionale di penalizzazione:
\begin{multline*}
J_{\bm \lambda }(f(\bm p,t))=\sum_{i=1}^n \sum_{j=1}^m \bigl( z_{ij} - f(\bm p_i,t_j) \bigr)^2 \ + \\
+\lambda_S  \sum_{k=1}^M \int_{\Omega} \Bigl( \Delta(  a_k(\bm p)  ) \Bigr)^2 d \bm p + \lambda_T \sum_{l=1}^N\int_{T_1}^{T_2} \Bigl( \frac{\partial^2   b_l(t)   }{\partial t ^2} \Bigr)^2 dt \ .
\end{multline*}
\end{frame}

\begin{frame}
Dati i vettori:
$$
\bm{c} =
\begin{bmatrix}
c_{11}  \\
\vdots\\
c_{1M}  \\
c_{21}  \\
\vdots\\
c_{NM}
\end{bmatrix}
\qquad
\bm \psi =
\begin{bmatrix}
\psi_{1}  \\
\psi_{2}  \\
\vdots\\
\psi_{N}
\end{bmatrix}
\qquad
\bm \psi_x=  \begin{bmatrix}
\partial \psi_{1}/\partial x \\
\partial \psi_{2}/\partial x  \\
\vdots\\
\partial \psi_{N}/\partial x \end{bmatrix} 
\qquad
\bm \psi_y=  \begin{bmatrix}
\partial \psi_{1}/\partial y  \\
\partial \psi_{2}/\partial y  \\
\vdots\\
\partial \psi_{N}/\partial y\end{bmatrix}
\qquad
\bm z =
\begin{bmatrix}
z_{11}  \\
\vdots\\
z_{1m}  \\
z_{21}  \\
\vdots\\
z_{nm}
\end{bmatrix}
$$
\ \ 
\newline
e le matrici: 
\begin{itemize}
\item $P_S=R_1 R_0^{-1} R_1 \qquad R_0 = \int_\Omega \bm \psi \bm \psi^T d \bm p \ , R_1 = \int_\Omega (\bm \psi_x \bm \psi_x^T + \bm \psi_y \bm \psi_y^T)d \bm p$
\item $P_T|_{k_1,k_2} =\int_{T_1}^{T_2} \varphi_{k_1}''(t) \varphi_{k_2}''(t)$
\item $P = \lambda_S\    (P_S \otimes I_M)   \ +\  \lambda_T\   (I_N \otimes P_T)$
\item $B = \Psi \otimes \Phi \qquad \Psi|_{i,l}=\psi_{l}(\bm p_i) \ , \Phi|_{j,k}=\varphi_{k}( t_j)$
\end{itemize}
Allora:
$$
J_{\bm \lambda }(\bm c) = (\bm z - B \bm c)^T (\bm z - B \bm c) + \bm c^T P \bm c \qquad \Rightarrow \qquad \hat  {\bm c} = (B^T B + P)^{-1}B^T \bm z
$$
\end{frame}


\subsection{Caso con covariate}
\begin{frame}
Si inseriscono nel modello $p$ possibili covariate:
$$
z_{ij}=\bm w_{ij}^T  \bm \beta + f(\bm{p}_i,t_j)+\varepsilon_{ij}\ \ \ \ i = 1,\ldots,n\ \ j=1,\ldots,m \ \ 
$$
Data la matrice disegno $W$, che si ottiene accostando per colonna i vettori di covariate, allora:
$$ J_{\bm \lambda }(\bm c) = (\bm z - W \bm \beta - B \bm c)^T (\bm z - W \bm \beta - B \bm c) + \bm c^t S \bm c $$
Derivando:
$$
\begin{cases}
W^TW \hat{\bm \beta} = W^T(\bm z - B \hat{\bm c})  \\
(B^T B + P) \hat{\bm c}=B^T(\bm z -W \hat{\bm \beta})
\end{cases} \qquad \Rightarrow \qquad
\begin{cases}
\hat{\bm \beta} = (W^TW)^{-1}W^T(\bm z - B \hat{\bm c}) \\
\hat  {\bm c} = AQ \bm z
\end{cases}
$$
con
$$
Q=[I-W(W^TW)^{-1}W^T] \qquad A=[B^TQB+P]^{-1}B^T
$$
\end{frame}

\section{Studi di simulazione}
\begin{frame}
\frametitle{Studi di simulazione}
\begin{columns}
	\begin{column}{0.6\textwidth}
	Le simulazioni sono state eseguite simulando da $f(\bm{p},t)=g(\bm{p})cos(t)$:
	del rumore:
	$$
	z_{ij}=f(\bm p_{i},t_j) + \beta w_{ij} + \varepsilon_{ij} \qquad \forall i , \forall j 
	$$
	dove:
	\begin{itemize}
	\item $\beta=1$ (eventuale)
	\item $w_{ij}\stackrel{\mathrm{iid}}{\sim}N(0,1) \qquad \forall i, \forall j$
	\item $\varepsilon_{ij}\stackrel{\mathrm{iid}}{\sim}N(0,0.5^2) \qquad \forall i, \forall j$
	\end{itemize} 
	I parametri di smoothing sono calcolati tramite GCV:
	\begin{itemize}
	\item senza covariata $\bm{\lambda}= (10^{-0.375}, 10^{-3.25})$
	\item con covariata $\bm{\lambda}= (10^{-0.5}, 10^{-3.25})$
	\end{itemize}
	\end{column}
	\begin{column}{0.4\textwidth}
		\begin{center}
		\includegraphics[width=0.8\textwidth]{Immagini/DomC_fstest.png}
		\end{center}
		\begin{center}
		\includegraphics[width=0.8\textwidth]{Immagini/DomC_Triangolazione.png}
		\end{center}
	\end{column}
\end{columns}
\end{frame}

\subsection{Caso senza covariata}
\begin{frame}
\begin{center}
\animategraphics[autoplay,loop,height=6cm]{1}{Immagini/Animazione/image}{1}{100}
\end{center}
\end{frame}

\subsection{Caso con covariata}
\begin{frame}
\begin{columns}
	\begin{column}{0.3\textwidth}
	Il modello stima:
	$$
	\hat{\beta} \approx 1.001 
	$$
	IC approssimato:
	$$
	\beta \in [0.9809;1.0225]
	$$
	\end{column}
	\begin{column}{0.6\textwidth}	
		\begin{flushright}		
		\animategraphics[autoplay,loop,height=5cm]{1}{Immagini/AnimazioneCovar/image}{1}{100}
		\end{flushright}
	\end{column}
\end{columns}
\end{frame}


\section{Confronto con altri metodi}
\subsection{Caso senza covariata}
\begin{frame}
\frametitle{Confronto con altri metodi}
\begin{columns}
	\begin{column}{0.6\textwidth}
	L'algoritmo è stato confrontato con altre tecniche già esistenti:
	\begin{itemize}
	\item KRIG: modello basato su kriging spazio-temporale
	\item TPS: basi in spazio \textit{Thin Plate Splines}, in tempo \textit{Smoothing Splines}
	\item SOAP: basi in spazio \textit{Soap Film Smoothing}, in tempo \textit{Smoothing Splines} 
	\end{itemize}
	Solo SOAP e ST-PDE possono tener conto del dominio spaziale!
	\end{column}
	\begin{column}{0.4\textwidth}
	\begin{flushright}
		\includegraphics[width=1\textwidth]{Immagini/Confronto_metodi.png}
	\end{flushright}		
	\end{column}
\end{columns}
\end{frame}

\begin{frame}
\begin{figure}
\centering
\begin{tabular}{lccccc}
& Reale & KRIG & TPS & SOAP & ST-PDE \\
\parbox[t]{2mm}{\multirow{3}{*}{\rotatebox[origin=r]{90}{$t=0$}}}&
\includegraphics[width=0.16\textwidth,valign=t]{Immagini/simulazioni/REALEtempo1.png} &
\includegraphics[width=0.16\textwidth,valign=t]{Immagini/simulazioni/KRIGtempo1.png}&
\includegraphics[width=0.16\textwidth,valign=t]{Immagini/simulazioni/TPStempo1.png}&
\includegraphics[width=0.16\textwidth,valign=t]{Immagini/simulazioni/SOAPtempo1.png}&
\includegraphics[width=0.16\textwidth,valign=t]{Immagini/simulazioni/STSRtempo1.png}\\
\parbox[t]{2mm}{\multirow{3}{*}{\rotatebox[origin=c]{90}{$t=\frac{\pi}{4}$}}}&
\includegraphics[width=0.16\textwidth,valign=t]{Immagini/simulazioni/REALEtempo2.png}&
\includegraphics[width=0.16\textwidth,valign=t]{Immagini/simulazioni/KRIGtempo2.png}&
\includegraphics[width=0.16\textwidth,valign=t]{Immagini/simulazioni/TPStempo2.png}&
\includegraphics[width=0.16\textwidth,valign=t]{Immagini/simulazioni/SOAPtempo2.png}&
\includegraphics[width=0.16\textwidth,valign=t]{Immagini/simulazioni/STSRtempo2.png}\\
\parbox[t]{2mm}{\multirow{3}{*}{\rotatebox[origin=c]{90}{$t=\frac{\pi}{2}$}}}&
\includegraphics[width=0.16\textwidth,valign=t]{Immagini/simulazioni/REALEtempo3.png}&
\includegraphics[width=0.16\textwidth,valign=t]{Immagini/simulazioni/KRIGtempo3.png}&
\includegraphics[width=0.16\textwidth,valign=t]{Immagini/simulazioni/TPStempo3.png}&
\includegraphics[width=0.16\textwidth,valign=t]{Immagini/simulazioni/SOAPtempo3.png}&
\includegraphics[width=0.16\textwidth,valign=t]{Immagini/simulazioni/STSRtempo3.png}\\
\parbox[t]{2mm}{\multirow{3}{*}{\rotatebox[origin=c]{90}{$t=\frac{3}{4}\pi$}}}&
\includegraphics[width=0.16\textwidth,valign=t]{Immagini/simulazioni/REALEtempo4.png}&
\includegraphics[width=0.16\textwidth,valign=t]{Immagini/simulazioni/KRIGtempo4.png}&
\includegraphics[width=0.16\textwidth,valign=t]{Immagini/simulazioni/TPStempo4.png}&
\includegraphics[width=0.16\textwidth,valign=t]{Immagini/simulazioni/SOAPtempo4.png}&
\includegraphics[width=0.16\textwidth,valign=t]{Immagini/simulazioni/STSRtempo4.png}
\end{tabular}
\end{figure}
\end{frame}

\section{Applicazione allo studio dei rifiuti nella provincia di Venezia}

\begin{frame}
\frametitle{Confronto con altri metodi}
\begin{columns}
	\begin{column}{0.6\textwidth}
	I dati sono stati localizzati in un unico punto, nel paese di riferimento del comune.
	\newline\newline
	Come covariata, per tener conto dell'effetto del turismo, si usa il numero di posti letto pro capite in strutture ricettive.
	\newline\newline
	Sono stati usati i valori pro capite, sia per il dato che per la covariata (possibilità di replicare i dati poichè densità).
	\end{column}
	\begin{column}{0.4\textwidth}
	\begin{center}
		\includegraphics[width=0.8\textwidth]{Immagini/Dati.png}
	\end{center}
	\begin{center}
		\includegraphics[width=0.8\textwidth]{Immagini/PL.png}
	\end{center}		
	\end{column}
\end{columns}
\end{frame}

\subsection{Caso senza covariata}
\begin{frame}
Risultati dell'applicazione del modello senza la covariata:
\newline
\begin{columns}
	\begin{column}{0.225\textwidth}
	\includegraphics[width=1\textwidth]{Immagini/Venezia/Cavarzere.png}
	\ \
	\newline
	\includegraphics[width=1\textwidth]{Immagini/Venezia/Cavallino-Treporti.png}
	\end{column}
	\begin{column}{0.225\textwidth}
	\includegraphics[width=1\textwidth]{Immagini/Venezia/Venezia.png}
	\ \
	\newline
	\includegraphics[width=1\textwidth]{Immagini/Venezia/Bibione.png}
	\end{column}
	\begin{column}{0.35\textwidth}	
		\begin{center}		
		\animategraphics[autoplay,loop,height=4.5cm,width=4cm]{1}{Immagini/Venezia/image}{1}{100}
		\end{center}
	\end{column}
\end{columns}
\end{frame}


\subsection{Caso con covariata}
\begin{frame}
Risultati dell'applicazione del modello con la covariata:
$$
\hat{\beta}\approx 30.5563 \qquad \beta \in [14.3158;46.7767]
$$
\begin{columns}
	\begin{column}{0.225\textwidth}
	\includegraphics[width=1\textwidth]{Immagini/VeneziaCovar/Cavarzere.png}
	\ \
	\newline
	\includegraphics[width=1\textwidth]{Immagini/VeneziaCovar/Cavallino-Treporti.png}
	\end{column}
	\begin{column}{0.225\textwidth}
	\includegraphics[width=1\textwidth]{Immagini/VeneziaCovar/Venezia.png}
	\ \
	\newline
	\includegraphics[width=1\textwidth]{Immagini/VeneziaCovar/Bibione.png}
	\end{column}
	\begin{column}{0.35\textwidth}	
		\begin{center}		
		\animategraphics[autoplay,loop,height=4.5cm,width=4cm]{1}{Immagini/VeneziaCovar/image}{1}{100}
		\end{center}
	\end{column}
\end{columns}
\end{frame}




\end{document}