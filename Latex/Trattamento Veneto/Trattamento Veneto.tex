\documentclass[a4paper,11pt,twoside,openright]{book}							% COMANDI INIZIALI
\usepackage[italian]{babel}								% sillabazione italiana
\usepackage[utf8]{inputenc}								% Per le lettere accentate IN UNIX E IN WINDOWS
\usepackage{ragged2e}					 				% giustifica
\usepackage{amsmath}									% Per allineare le equazioni
\usepackage{amssymb}									% Per le lettere dell'indicatrice (mathbb)

\justifying 										% giustifica

\date{28 Luglio 2014}
\author{Gabriele Mazza}
\title{Dimostrazione $S_{time}$ e $S_{space}$}

\begin{document}

%Indice e numerazione
\pagenumbering{arabic}

\chapter{I dati dei rifiuti nel Veneto}


L'applicazione scelta per lo studio del modello riguarda i dati di analisi di produzione di rifiuti urbani nel periodo di anni dal 1997 al 2011 in Veneto. Per rifiuti urbani si intendono rifiuti domestici, prodotti in locali, aree pubbliche, parchi o giardini, spiagge o provenienti dalla pulitura delle strade o di altri luoghi pubblici. Non sono conteggiati i rifiuti speciali (tra cui ad esempio gli industriali, agricoli o provenienti da attività commerciali o di costruzione) o pericolosi (per i quali esistono programmi di smaltimento particolari).

I dati sono stati raccolti e pubblicati dall'Agenzia regionale per la prevenzione e protezione ambientale del Veneto (Arpav) e sono disponibili alla consultazione e al trattamento.

Per ogni comune del Veneto e per ogni anno è disponibile il numero di rifiuti totali raccolti in tonnellate. Inoltre, per ogni anno è indicata la popolazione di ogni comune del Veneto. Perciò la quantità di riferimento non sarà il valore dei rifiuti totali raccolti in ogni anno per comune, ma il valore pro capite, in quanto la popolazione influenza in maniera considerevole la produzione di rifiuti.

Le coordinate spaziali dei comuni sono la longitudine e la latitudine.

\section{La scelta delle covariate}

La produzione annua di rifiuti nei comuni può essere influenzata da alcuni fattori, e il più importante di tutti è certamente la popolazione. Una parte molto importante è rappresentata dalla popolazione residente, che è già stata inclusa nella risposta, poiché per uniformità i rifiuti sono studiati come valore pro capite nel comune. Tuttavia anche i turisti sono una componente non trascurabile di produzione di rifiuti urbani.

Il Veneto ha molte zone di elevata attrazione turistica. La più importante di queste è Venezia, ma si hanno anche zone balneari (come... AGGIUNGI). L'informazione scelta per sintetizzare l'attività turistica è il numero di posti letto presente sul territorio, valore disponibile grazie all'applicativo dell'Istat \textit{Atlante Statistico dei Comuni} ad ogni anno a livello comunale. Il totale dei posti letto per comune è la somma di vari tipi di attività, non solamente alberghiere (ad esempio sono conteggiati anche esercizi complementari, bed \& breakfast, campeggi) e saranno considerati normalizzati per la popolazione residente per uniformità con la risposta.

\section{Il trattamento del territorio}

Per poter studiare il problema a livello computazionale occorre avere una buona approssimazione della frontiera della regione. Questa è disponibile nel pacchetto di R (SPECIFICARE...) che descrive la regione come unione di N poligoni distinti (a causa delle numerose isole di cui è composta la laguna) composti da un alto numero di vertici. Non è possibile analizzare il problema su un territorio così descritto, perciò è stata necessaria una analisi iniziale della frontiera per ridurne la complessità.

Inanzitutto sono stati scelti solo alcuni poligoni della regione, corrispondenti al Veneto e alle isole di Venezia, Murano, Lido e Pellestrina (più rilevanti a livello di popolazione e turismo). Di queste isole occorreva una buona tecnica di riduzione del numero di punti di frontiera, per avere una trattazione computazionale più efficiente.

Per risolvere questo problema si è scelto di ricorrere ad un'analisi di smoothing della frontiera di ogni poligono. Una volta ricavata l'ascissa curvilinea dei poligoni, tramite \textit{Regression Splines} cubiche sono state analizzate due funzioni: ascissa cuSono state considerate due di dati funzio

































\end{document}